\documentclass[spanish, 10pt,a4paper]{article}
\usepackage[spanish]{babel}
\usepackage[utf8]{inputenc}
\usepackage{textcomp}
\usepackage{hyperref}
\usepackage[pdftex]{graphicx}
\usepackage{epsfig}
\usepackage{amsmath}
\usepackage{hyperref}
\usepackage{amssymb}
\usepackage{color}
\usepackage{graphics}
\usepackage{amsthm}
\usepackage{subcaption}
\usepackage{caratula}
\usepackage{fancyhdr,lastpage}
\usepackage[paper=a4paper, left=1.4cm, right=1.4cm, bottom=1.4cm, top=1.4cm]{geometry}
\usepackage[table]{xcolor} % color en las matrices
\usepackage[font=small,labelfont=bf]{caption} % caption de las figuras en letra mas chica que el texto
\usepackage{listings}
\usepackage{float}
\usepackage{pdfpages}
\usepackage{amsfonts}


\color{black}

%%%PAGE LAYOUT%%%
\topmargin = -1.2cm
\voffset = 0cm
\hoffset = 0em
\textwidth = 48em
\textheight = 164 ex
\oddsidemargin = 0.5 em
\parindent = 2 em
\parskip = 3 pt
\footskip = 7ex
\headheight = 20pt
\pagestyle{fancy}
\lhead{PLP - Trabajo Pr\'actico 1} % cambia la parte izquierda del encabezado
\renewcommand{\sectionmark}[1]{\markboth{#1}{}} % cambia la parte derecha del encabezado
\rfoot{\thepage}
\cfoot{}
\numberwithin{equation}{section} %sets equation numbers <chapter>.<section>.<subsection>.<index>

\newcommand{\figurewidth}{1\textwidth}

\newcommand{\tuple}[1]{\ensuremath{\left \langle #1 \right \rangle }}
\newcommand{\Ode}[1]{\small{$\mathcal{O}(#1)$}}


%El siguiente paquete permite escribir la caratula facilmente
\hypersetup{
  pdftitle={ PLP - TP1 },
  colorlinks,
  citecolor=black,
  filecolor=black,
  linkcolor=black,
  urlcolor=black 
}

\materia{Paradigmas de Lenguajes}

\titulo{Trabajo Práctico 1}

\subtitulo{Paradigma Funcional.}

\grupo{Grupo Two and a Half Blondes}

\integrante{De Sousa Bispo, Germán}{359/12}{germandesousa@gmail.com}
\integrante{Fernandez, Esteban}{691/12}{esteban.pmf@gmail.com}
\integrante{Wright, Carolina}{876/12}{wright.carolina@gmail.com}

 
\begin{document}
{ \oddsidemargin = 2em
	\headheight = -20pt
	\maketitle
}
	\tableofcontents
	\newpage

\section{Código}

\subsection{Ejercicio 1}
\begin{lstlisting}
split :: Eq a => a -> [a] -> [[a]]
split elementoSeparador xs = filter (\word -> length word > 0) (foldr f [[]] xs) 
		             where f = (\x xss -> if x == elementoSeparador 
						then [] : xss 
						else (x : head(xss)) : tail(xss))
\end{lstlisting}

\subsection{Ejercicio 2}

\begin{lstlisting}
longitudPromedioPalabras :: Extractor
longitudPromedioPalabras xs =  mean (map genericLength (listaDePalabras xs)) 
			       where listaDePalabras xs = split ' ' xs
\end{lstlisting}

\subsection{Ejercicio 3}

\begin{lstlisting}
cuentas :: Eq a => [a] -> [(Int, a)]
cuentas xs = zip cantidadDeRepeticiones (nub xs) 
	where cantidadDeRepeticiones = [length (filter (== y) xs)| y <- nub xs]
\end{lstlisting}

\subsection{Ejercicio 4}

\begin{lstlisting}
repeticionesPromedio :: Extractor
repeticionesPromedio xs = mean ( map (\tupla -> fromIntegral (fst tupla)) 
				     (cuentas (listaDePalabras xs))) 
			  where listaDePalabras xs = split ' ' xs 

tokens :: [Char]
tokens = "\_,)(*;-=>/.{}\"&:+#[]<|\%!\'@?~^$` abcdefghijklmnopqrstuvwxyz
	 0123456789"
\end{lstlisting}

\subsection{Ejercicio 5}

\begin{lstlisting}
fstDeLaUnicaTuplaEnLista :: [(Int, a)] -> Int
fstDeLaUnicaTuplaEnLista [] = 0
fstDeLaUnicaTuplaEnLista [x] = fst x 

frecuenciaTokens :: [Extractor]
frecuenciaTokens =  map (\token -> (\texto -> let elemNQuantities = 
			cuentas texto in 
			(getTokenQuantityIn elemNQuantities token) /
			(sumAllQuantitiesIn elemNQuantities))) tokens 

sumAllQuantitiesIn :: [(Int, a)] -> Float
sumAllQuantitiesIn = (\elemNQuantities -> fromIntegral 
				$ sum  
			    	$ map (\elemAndQuantity -> fst elemAndQuantity)
			    	$ elemNQuantities)

getTokenQuantityIn :: Eq a => [(Int, a)] -> a -> Float
getTokenQuantityIn = (\elemNQuantities token -> fromIntegral 
		     $ fstDeLaUnicaTuplaEnLista 
		     $ filter (\elemAndQuantity -> (snd elemAndQuantity)==token) 
		     $ elemNQuantities)
\end{lstlisting}


\subsection{Ejercicio 6}

\begin{lstlisting}
normalizarExtractor :: [Texto] -> Extractor -> Extractor
normalizarExtractor [] extractor = const 0
normalizarExtractor textos extractor =  let maximoFeature = 
			maximum (map abs [(extractor texto) | texto <- textos]) 
			in (\text -> (extractor text) / maximoFeature)
\end{lstlisting}


\subsection{Ejercicio 7}

\begin{lstlisting}
extraerFeatures :: [Extractor] -> [Texto] -> Datos 
extraerFeatures extractores textos = let extractoresNorm = 
		map (\extr -> normalizarExtractor textos extr) extractores 
		in map (\text-> 
		    (map (\normExtr -> normExtr text) extractoresNorm)) textos
\end{lstlisting}


\subsection{Ejercicio 8}

\begin{lstlisting}
distEuclideana :: Medida
distEuclideana p q = sqrt (sum (binomiosCuadrado p q) ) 
		 where binomiosCuadrado p q = map (\x -> x*x) (zipWith (-) p q)

distCoseno :: Medida
distCoseno p q = (sumatoriaProductos p q) / (productoVectorial p q)

sumatoriaProductos :: Medida
sumatoriaProductos p q = sum (productos p q)
		       where productos = zipWith (*)

productoVectorial :: Medida
productoVectorial p q = sqrt((sumatoriaProductos p p)*(sumatoriaProductos q q))
\end{lstlisting}


\subsection{Ejercicio 9}

\begin{lstlisting}
knn :: Int -> Datos -> [Etiqueta] -> Medida -> Modelo
knn n matrizDatos etiquetas fDistancia = (\instancia -> moda n etiquetas 
    $ zip (getDistanciasAInstancia matrizDatos instancia fDistancia) etiquetas)

getDistanciasAInstancia :: Datos -> Instancia -> Medida -> [Float]
getDistanciasAInstancia = (\matrizDatos instancia fDistancia ->
				map (\dato -> 
					fDistancia dato instancia) matrizDatos) 

moda :: Int  -> [Etiqueta] -> [(Float, Etiqueta)] -> Etiqueta
moda = (\n etiquetas distsConEtiqs  -> snd $ maximumBy compare 
				  $ getNMejoresEtiqs n distsConEtiqs etiquetas)

getNMejoresEtiqs :: Int->[(Float, Etiqueta)]->[Etiqueta] ->[(Int, Etiqueta)]
getNMejoresEtiqs = (\n distanciasConEtiqs etiquetas -> 
			contarAparicionesEtiq (nub etiquetas) 
			$ take n 
			$ sortBy compare distanciasConEtiqs)

contarAparicionesEtiq :: [Etiqueta] -> [(Float, Etiqueta)] -> [(Int, Etiqueta)]
contarAparicionesEtiq = (\etiquetasSinRepe nMasCercanos ->
		 [(aparicionesEtiqueta, etiqueta) | etiqueta<-etiquetasSinRepe, 
			let aparicionesEtiqueta = 
			 length(filter(matcheaEtiqueta etiqueta) nMasCercanos)])

matcheaEtiqueta:: Etiqueta -> (Float, Etiqueta) -> Bool
matcheaEtiqueta etiqueta = (\label tupla -> label==(snd tupla)) etiqueta
\end{lstlisting}


\subsection{Ejercicio 10}
\begin{lstlisting}
separarDatos :: Datos -> [Etiqueta] -> Int -> Int 
		-> (Datos, Datos, [Etiqueta], [Etiqueta])
separarDatos datos etiquetas n p = 
   let datosParticionado = (sacarInvalidos (foldl (\z elem -> 
	  if (length (last z)) < (div (length datos) n) 
	  then (init z) ++ [(last z) ++ [elem]] 
	  else (z ++ [[elem]])) [[]] datos) n) 
   in let etiquetasParticionado = (sacarInvalidos (foldl (\z elem -> 
	  if (length (last z)) < (div (length etiquetas) n) 
	  then (init z) ++ [(last z) ++ [elem]] 
	  else (z ++ [[elem]])) [[]] etiquetas) n) 
   in (getTrain datosParticionado p, getVal datosParticionado p, 
       getTrain etiquetasParticionado p, getVal etiquetasParticionado p)

sacarInvalidos :: [[a]] -> Int -> [[a]]
sacarInvalidos datos n = if (length (last datos)) < n 
			 then init datos else datos

getTrain:: [[a]] -> Int -> [a]
getTrain datos p = concat ((take (p-1) datos) ++ (drop p datos))

getVal::[a] -> Int -> a
getVal datos n = last (take n datos)
\end{lstlisting}

\subsection{Ejercicio 11}
\begin{lstlisting}
accuracy :: [Etiqueta] -> [Etiqueta] -> Float
accuracy e1 e2 = sumaIguales (zip e1 e2) / fromIntegral (length (zip e1 e2)) 
   where sumaIguales = foldr (\t rec -> if fst t == snd t 
					then 1+rec 
					else rec) 0
\end{lstlisting}

\subsection{Ejercicio 12}
\begin{lstlisting}
nFoldCrossValidation :: Int -> Datos -> [Etiqueta] -> Float
nFoldCrossValidation n datos etiquetas = mean $ accuracyN 
	$ applyKnnToPartitions [separarDatos datos etiquetas n p | p <- [1..n]]

applyKnnToPartitions :: [(Datos, Datos, [Etiqueta], [Etiqueta])] 
			-> [([Etiqueta], [Etiqueta])]
applyKnnToPartitions = map (\(xTrain, xValid, yTrain, yValid) -> 
			      (applyKnnToAllValid xTrain yTrain xValid, yValid))  

applyKnnToAllValid :: Datos -> [Etiqueta] -> Datos -> [Etiqueta]
applyKnnToAllValid = (\xTrain yTrain xValid -> 
  let trainedKnn = knn 15 xTrain yTrain distEuclideana 
		    in map (\validInstancia -> trainedKnn validInstancia)xValid)

accuracyN :: [([Etiqueta], [Etiqueta])] -> [Float]
accuracyN = map (\(etiquetasSupuestas, etiquetasReales) -> 
			accuracy etiquetasSupuestas etiquetasReales) 
\end{lstlisting}

\section{Tests}
\subsection{Ejercicio 1}
\begin{lstlisting}
splitPorEspacioPresente = split ' ' "Habia una vez."
splitPorComaPresente = split ',' "Habia, una, vez."
splitPorEspacioNoPresente = split ' ' "Habiaunavez."
splitPorComaNoPresente = split ',' "Habia una vez."


splitTest1 = TestCase (assertEqual "Por espacio, presente" 
			  ["Habia", "una", "vez."] splitPorEspacioPresente)
splitTest2 = TestCase (assertEqual "Por coma, presente" 
			  ["Habia", " una", " vez."] splitPorComaPresente)
splitTest3 = TestCase (assertEqual "Por espacio, no presente " 
			  ["Habiaunavez."] splitPorEspacioNoPresente)
splitTest4 = TestCase (assertEqual "Por coma, no presente" 
			  ["Habia una vez."] splitPorComaNoPresente)
\end{lstlisting}

\subsection{Ejercicio 2}
\begin{lstlisting}
longitudPromedioLetrasSueltas = longitudPromedioPalabras "a b c d e f g h i"
longitudPromedioDosLetrasXPalabra = longitudPromedioPalabras 
				   "aa bb cc dd ee ff gg hh ii"
longitudPromedioUnaPalabraLarga = longitudPromedioPalabras 
				   "aabbccddeeffgghhii"
longitudPromedioDiferentesTamanios2Palabras = longitudPromedioPalabras 
				   "aabbcc aabb"
longitudPromedioDiferentesTamanios3Palabras = longitudPromedioPalabras 
				   "aabbcc aabb ad"


longitudPromedioTest1 = TestCase (assertEqual "Letras sueltas" 
				1 longitudPromedioLetrasSueltas)
longitudPromedioTest2 = TestCase (assertEqual "Dos letras por palabra" 
				2 longitudPromedioDosLetrasXPalabra)
longitudPromedioTest3 = TestCase (assertEqual "Una palabra larga" 
				18 longitudPromedioUnaPalabraLarga)
longitudPromedioTest4 = TestCase (assertEqual "Diferentes tamanios, 2 palabras"
				5 longitudPromedioDiferentesTamanios2Palabras)
longitudPromedioTest5 = TestCase (assertEqual "Diferentes tamanios, 3 palabras" 
				4 longitudPromedioDiferentesTamanios3Palabras)
\end{lstlisting}

\subsection{Ejercicio 3}
\begin{lstlisting}
cuentasDelVacio = cuentas [""]
cuentasDelEspacio = cuentas [" "]
cuentasUnaPalabraUnaRepeticion = cuentas ["Una"]
cuentasUnaPalabraVariasRepeticiones = cuentas ["Una", "Una", "Una"]
cuentasMuchasPalabrasUnaRepeticion = cuentas  ["Una", "Dos", "Tres", "Cuatro"]
cuentasMuchasPalabrasMuchasRepeticiones = cuentas  ["Una", "Dos", "Tres", "Cuatro", 
						    "Una", "Dos", "Tres", "Cuatro",  
						    "Una", "Dos", "Tres", "Cuatro"]

cuentasTest1 = TestCase (assertEqual "Vacio" [(1,"")] cuentasDelVacio)
cuentasTest2 = TestCase (assertEqual "Espacio" [(1," ")] cuentasDelEspacio)
cuentasTest3 = TestCase (assertEqual "Una palabra una repeticion" 
				[(1,"Una")] cuentasUnaPalabraUnaRepeticion)
cuentasTest4 = TestCase (assertEqual "Una palabra varias repeticiones" 
				[(3,"Una")] cuentasUnaPalabraVariasRepeticiones)
cuentasTest5 = TestCase (assertEqual "Muchas palabras, una repeticion" 
				[(1,"Una"),(1,"Dos"),(1,"Tres"),(1,"Cuatro")] 
				cuentasMuchasPalabrasUnaRepeticion)
cuentasTest6 = TestCase (assertEqual "Muchas palabras, muchas repeticiones" 
				[(3,"Una"),(3,"Dos"),(3,"Tres"),(3,"Cuatro")] 
				cuentasMuchasPalabrasMuchasRepeticiones)

\end{lstlisting}

\subsection{Ejercicio 4}
\begin{lstlisting}
repeticionesPromedioVacio = repeticionesPromedio ""
repeticionesPromedioUnaPalabra = repeticionesPromedio "Una"
repeticionesPromedioUnaPalabra3Veces = repeticionesPromedio "Una Una Una"
repeticionesPromedioMuchasPalabras1Vez = 
				repeticionesPromedio  "Una Dos Tres Cuatro"
repeticionesPromedioMuchasPalabras3Veces = 
				repeticionesPromedio  "Una Dos Tres Cuatro 
						       Una Dos Tres Cuatro
						       Una Dos Tres Cuatro"

repeticionesPromedioTest3 = TestCase (assertEqual 
			   	      "Una palabra una repeticion"  
				      1 repeticionesPromedioUnaPalabra)
repeticionesPromedioTest4 = TestCase (assertEqual 
 				      "Una palabra varias repeticiones" 
				      3 repeticionesPromedioUnaPalabra3Veces)
repeticionesPromedioTest5 = TestCase (assertEqual 
			 	      "Muchas palabras, una repeticion" 
				      1 repeticionesPromedioMuchasPalabras1Vez)
repeticionesPromedioTest6 = TestCase (assertEqual 
				      "Muchas palabras, muchas repeticiones" 
				      3 repeticionesPromedioMuchasPalabras3Veces)

\end{lstlisting}

\subsection{Ejercicio 5}
\begin{lstlisting}
frecuenciasTokensElToken = (head frecuenciaTokens) "_"
frecuenciasTokensUnaPalabraSinToken = (head frecuenciaTokens) "Una"
frecuenciasTokensUnaPalabraConToken = (head frecuenciaTokens) "Una_"
frecuenciasTokensMuchasPalabrasSinToken = 
				(head frecuenciaTokens)  "Una Dos Tres Cuatro"
frecuenciasTokensPalabrasConToken = 
				(head frecuenciaTokens) "Una_Dos_Tres"

frecuenciasTokensTest2 = TestCase (assertEqual "Solo el Token" 
				   1 frecuenciasTokensElToken)
frecuenciasTokensTest3 = TestCase (assertEqual "Una palabra sin Token" 
				   0 frecuenciasTokensUnaPalabraSinToken)
frecuenciasTokensTest4 = TestCase (assertEqual "Una palabra con Token" 
				   0.25 frecuenciasTokensUnaPalabraConToken)
frecuenciasTokensTest5 = TestCase (assertEqual "Muchas palabras, sin Token" 
				   0 frecuenciasTokensMuchasPalabrasSinToken)
frecuenciasTokensTest6 = TestCase (assertEqual "Muchas palabras, con Token"  
				  (formatFloatN 0.1666667 3) 
				  (formatFloatN 
					frecuenciasTokensPalabrasConToken 3))
\end{lstlisting}

\subsection{Ejercicio 6 y Ejercicio 7}
\begin{lstlisting}
checkNormalizado :: [Float] -> Bool
checkNormalizado xs = ((head xs)) >= 0 && ((head xs) <= 1) 
		       && ((head (tail xs)) <= 1) &&  ((head (tail xs)) >= 0)

estaTodoNormalizado :: [[Float]] -> Bool
estaTodoNormalizado xss = foldr (\parDeFeatures -> 
			  \rec -> 
			      (checkNormalizado parDeFeatures) && rec) True xss

aCheckearNormalizado = 
	extraerFeatures [longitudPromedioPalabras, repeticionesPromedio]
			["b=a", "a = 2; a = 4", "asd", "1233243453", 
			"assadasdasasd", "123 as"]

normalizarExtractorTest1 = TestCase (assertEqual "Esta normalizado" 
			   True (estaTodoNormalizado aCheckearNormalizado))

\end{lstlisting}


\subsection{Ejercicio 8}
\begin{lstlisting}
distanciaEuclideanaCero = distEuclideana [0,0] [0,0]
distanciaEuclideana2 = distEuclideana [1,0] [0,1]
distanciaEuclideana4 = distEuclideana [1,1] [1,1]

distanciaCosenoCero = distEuclideana [0,0] [0,0]
distanciaCoseno2 = distEuclideana [1,0] [0,1]
distanciaCoseno4 = distEuclideana [1,1] [1,1]

distanciaEuclideanaTest1 = TestCase (assertEqual "DistEuclideana Ceros" 
				    0 distanciaEuclideanaCero)
distanciaEuclideanaTest2 = TestCase (assertEqual "DistEuclideana 1 0, 0 1" 
				    (formatFloatN 1.414214 3) 
				    (formatFloatN distanciaEuclideana2 3))
distanciaEuclideanaTest3 = TestCase (assertEqual "DistEuclideana 1 1, 1 1" 
				    0 distanciaEuclideana4)

distanciaCosenoTest1 = TestCase (assertEqual "DistCoseno Ceros" 
				0 distanciaCosenoCero)
distanciaCosenoTest2 = TestCase (assertEqual "DistCoseno 1 0, 0 1" 
				(formatFloatN 1.414214 3) 
				(formatFloatN distanciaCoseno2 3))
distanciaCosenoTest3 = TestCase (assertEqual "DistCoseno 1 1, 1 1" 
				0 distanciaCoseno4)
\end{lstlisting}

\subsection{Ejercicio 9}
\begin{lstlisting}
knnEnun = (knn 2 [[0,1],[0,2],[2,1],[1,1],[2,3]] 
	       ["i","i","f","f","i"] distEuclideana) [1,1]
knnTest = TestCase (assertEqual "Knn Enunciado" "f" knnEnun)
\end{lstlisting}

\subsection{Ejercicio 10}
\begin{lstlisting}
xsTestEj10 = [[1,1],[2,2],[3,3],[4,4],[5,5],[6,6],[7,7]] :: Datos
yTestEj10 = ["1","2","3","4","5","6","7"]
(x_train, x_val, y_train, y_val) = separarDatos xsTestEj10 yTestEj10 3 2

separarDatosTest = TestCase (assertEqual "SepararDatos" 
			    (x_train, y_train) 
			    ([[1.0,1.0],[2.0,2.0],[5.0,5.0],[6.0,6.0]],
			     ["1","2","5","6"]))
\end{lstlisting}


\subsection{Ejercicio 11}
\begin{lstlisting}
accuracy0 = accuracy ["i", "i", "i", "i", "i"] ["f", "f", "f", "f", "f"]
accuracy60 = accuracy ["f", "f", "i", "i", "f"] ["i", "f", "i", "f", "f"]
accuracy100 = accuracy ["f", "f", "f", "f", "f"] ["f", "f", "f", "f", "f"]

accuracyTest1 = TestCase (assertEqual "0\% accuracy" 0 accuracy0)
accuracyTest2 = TestCase (assertEqual "60\% accuracy" 0.6 accuracy60)
accuracyTest3 = TestCase (assertEqual "100\% accuracy" 1 accuracy100)
\end{lstlisting}

\subsection{Ejercicio 12}
\begin{lstlisting}
twoFoldValidation = nFoldCrossValidation 2 
				      [[1,1],[2,2],[3,3],[4,4],[5,5],
				       [6,6],[7,7],[8,8],[9,9],[10,10]]
				      ["i","f","f","i","i","i","i","i","i","i"]
threeFoldValidation = nFoldCrossValidation 3 
				      [[1,1],[2,2],[3,3],[4,4],[5,5],
				       [6,6],[7,7],[8,8],[9,9],[10,10]] 
				      ["i","f","f","i","i","i","i","i","i","i"]
fourFoldValidation = nFoldCrossValidation 4 
				      [[1,1],[2,2],[3,3],[4,4],[5,5],
				       [6,6],[7,7],[8,8],[9,9],[10,10]] 
				      ["i","f","f","i","i","i","i","i","i","i"]

test2FoldValidation = TestCase (assertEqual "2FoldValidation" 
				0.8 twoFoldValidation)
test3FoldValidation = TestCase (assertEqual "3FoldValidation" 
				(formatFloatN 0.7777777 3) 
				(formatFloatN threeFoldValidation 3))
test4FoldValidation = TestCase (assertEqual "4FoldValidation" 
				0.75 fourFoldValidation)
\end{lstlisting}
	%~ \newpage
	%~ \bibliographystyle{plain}
	%~ \clearpage
	%~ \bibliography{bibliography}
	%~ \addcontentsline{toc}{section}{Referencias}

\end{document}
