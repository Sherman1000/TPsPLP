\documentclass[spanish, 10pt,a4paper]{article}
\usepackage[spanish]{babel}
\usepackage[utf8]{inputenc}
\usepackage{textcomp}
\usepackage{hyperref}
\usepackage[pdftex]{graphicx}
\usepackage{epsfig}
\usepackage{amsmath}
\usepackage{hyperref}
\usepackage{amssymb}
\usepackage{color}
\usepackage{graphics}
\usepackage{amsthm}
\usepackage{subcaption}
\usepackage{caratula}
\usepackage{fancyhdr,lastpage}
\usepackage[paper=a4paper, left=1.4cm, right=1.4cm, bottom=1.4cm, top=1.4cm]{geometry}
\usepackage[table]{xcolor} % color en las matrices
\usepackage[font=small,labelfont=bf]{caption} % caption de las figuras en letra mas chica que el texto
\usepackage{listings}
\usepackage{float}
\usepackage{pdfpages}
\usepackage{amsfonts}


\color{black}

%%%PAGE LAYOUT%%%
\topmargin = -1.2cm
\voffset = 0cm
\hoffset = 0em
\textwidth = 48em
\textheight = 164 ex
\oddsidemargin = 0.5 em
\parindent = 2 em
\parskip = 3 pt
\footskip = 7ex
\headheight = 20pt
\pagestyle{fancy}
\lhead{PLP - Recuperatorio Trabajo Pr\'actico 2} % cambia la parte izquierda del encabezado
\renewcommand{\sectionmark}[1]{\markboth{#1}{}} % cambia la parte derecha del encabezado
\rfoot{\thepage}
\cfoot{}
\numberwithin{equation}{section} %sets equation numbers <chapter>.<section>.<subsection>.<index>

\newcommand{\figurewidth}{1\textwidth}

\newcommand{\tuple}[1]{\ensuremath{\left \langle #1 \right \rangle }}
\newcommand{\Ode}[1]{\small{$\mathcal{O}(#1)$}}


%El siguiente paquete permite escribir la caratula facilmente
\hypersetup{
  pdftitle={ PLP - TP1 },
  colorlinks,
  citecolor=black,
  filecolor=black,
  linkcolor=black,
  urlcolor=black 
}

\materia{Paradigmas de Lenguajes}

\titulo{Trabajo Práctico 1}

\subtitulo{Paradigma Funcional.}

\grupo{Grupo Two and a Half Blondes}

\integrante{De Sousa Bispo, Germán}{359/12}{germandesousa@gmail.com}
\integrante{Fernandez, Esteban}{691/12}{esteban.pmf@gmail.com}
\integrante{Wright, Carolina}{876/12}{wright.carolina@gmail.com}

 
\begin{document}
{ \oddsidemargin = 2em
	\headheight = -20pt
	\maketitle
}
	\tableofcontents
	\newpage

\section{Correcciones sobre primera entrega}
Se realizaron principalmente cuatro modificaciones. 
\begin{itemize}
\item asignar\_var fue modificada para no utilizar listas intermedias y aprovechar \textit{member}.
\item Se rehizo descifrar para realizar la unificación a nivel de lista y no de elemento como estaba en el TP anterior. Además, recordemos que esta operación no funcionaba en la primera entrega. Se utilizó cant\_distintos para evitar la unificación de una letra con más de un símbolo y de uno solo de estos con una letra.
\item Se reemplazó la implementación de split\_por\_caracter por una que aprovechase de \textit{append} a partir de \textit{Generate} \& \textit{Test}.
\item Se modificó levemente con\_espacios\_intercalados para evitar usar un cut innecesario en intercalar\_o\_no. Básicamente genera toda combinación de espacios aprovechándonos de la forma de ejecución de Prolog.
\end{itemize}

\section{Código}

\subsection{Ejercicio 1}

\begin{itemize}
\item diccionario\_lista(?Lcode): la reversibilidad depende de la reversibilidad de string\_codes, la cual es string\_codes(?String, ?Codes).
\end{itemize}
\begin{lstlisting}
diccionario_lista(Lcode) :- diccionario(PalabraDelDicc), 
			    string_codes(PalabraDelDicc, Lcode). 
\end{lstlisting}

\subsection{Ejercicio 2}

\begin{itemize}
\item juntar\_con(+S, ?J , ?R): Si no se instancia S, el predicado no funciona correctamente (aunque no se cuelga) por la forma en que se realizan los Appends. 
J puede ser una variable que, si R está instanciado, terminará unificándose. Si no, appendeará a la variable J.
\end{itemize}
\begin{lstlisting}
juntar_con([L], _ , L).
juntar_con([Ls | Lss], J, R) :- juntar_con(Lss, J, Rrec), 
				append(RPref, Rrec, R), 
				append(Ls, [J], RPref),!.  
\end{lstlisting}

\subsection{Ejercicio 3}

\begin{itemize}
\item palabras(+S, ?P): Los motivos por los que S debe estar instanciado se muestran en el análisis de split\_por\_caracter. El mismo se encuentra en el ejercicio 10.
\end{itemize}
\begin{lstlisting}
palabras(S, P) :- split_por_caracter(S, espacio, P).  
\end{lstlisting}

\subsection{Ejercicio 4}

\begin{itemize}
\item asignar\_var(?A, +MI, ?MF): Si MI no está instanciado, se cuelga cuando trata de encontrar otra solución más que MI = [] y MF = A, debido a que genera
combinaciones con listas infinitas. A o MF deben estar instanciados. No pueden no estar instanciados al mismo tiempo.

\item Este ejercicio funciona gracias a la capacidad de ProLog de generar variables frescas bajo demanda utilizando 
variables anonimas generadas con la keyword "\_", ademas, la manera de representar dichas variables frescas ayuda para que estas puedan ser manipulables dentro de los predicados.
Si se nos diera como variable fresca una que ya hemos utilizado antes nuestro predicado asignar\_var se volveria inconsistente.
\end{itemize}
\begin{lstlisting}

asignar_var(A, MI, MF) :- not(member((A, _), MI)), 
			  append(MI, [(A, _)], MF).
asignar_var(A, MI, MI) :- member((A, _), MI).

\end{lstlisting}

\subsection{Ejercicio 5}

\begin{itemize}
\item palabras\_con\_variables(+L, ?V): Si L no estuviera instanciado, no se rompe pero genera infinitos resultados inservibles con combinaciones de listas vacias. Esto sucede debido a la utilización de member().
\item palabras\_con\_var\_y\_mapa(+L, ?V, +Mi, -Mf)
\item pares\_definidos\_en\_mapa(+L, ?V, +Mi, -Mf)
\end{itemize}
\begin{lstlisting}
palabras_con_variables(L, V) :- palabras_con_var_y_mapa(L, V, [], _).

palabras_con_var_y_mapa([], [], Mf, Mf).
palabras_con_var_y_mapa([Ls | Lss], [Vs | Vss], M, Mf1) :- 
				pares_definidos_en_mapa(Ls, Vs, M, Mf0), 
				palabras_con_var_y_mapa(Lss, Vss, Mf0, Mf1).

pares_definidos_en_mapa([], [], Mf, Mf).
pares_definidos_en_mapa([X | Ls], [V | Vs], Mi0, Mf) :-
				asignar_var(X, Mi0 , Mi1), 
				member((X,V), Mi1), 
				pares_definidos_en_mapa(Ls, Vs, Mi1, Mf).
\end{lstlisting}


\subsection{Ejercicio 6}

\begin{itemize}
\item quitar(?E, +Ls, ?R): E puede no estar instanciado ya que el algoritmo también elimina variables en caso de estar. Ls deberá estar instanciada para evitar la generación de 
listas infinitas. Si no esta instanciado, devuelve una solución trivial y luego se cuelga cuando se pide otra (debido al uso de append).
\end{itemize}
\begin{lstlisting}
quitar(_, [], R) :- R = [].
quitar(E, [L|Ls], R) :- E == L, quitar(E, Ls, R).
quitar(E, [L|Ls], R) :- E \== L, quitar(E, Ls, Rrec), 
			append([L], Rrec, R).
\end{lstlisting}


\subsection{Ejercicio 7}

\begin{itemize}
\item cant\_distintos(+Ls, ?S): Ls deberá estar instanciado para evitar la generación de listas infinitas. Si no esta instanciado, devuelve una solución trivial y luego se cuelga 
cuando se pide otra (debido al uso de quitar(), quien termina llamando a append()).
\end{itemize}
\begin{lstlisting}
cant_distintos([], S) :- S = 0.
cant_distintos([L | Ls], S) :- quitar(L, Ls, RQuitado), 
			       cant_distintos(RQuitado, CuentaRec), 
			        S is (1 + CuentaRec).
\end{lstlisting}


\subsection{Ejercicio 8}

\begin{itemize}
\item descifrar(+S, ?M): S deberá estar instanciado. M podrá no estar instanciado, por lo que devolverá el resultado de descifrar, o puede estar instanciado y retornará verdadero o falso. Básicamente sucede debido a la instanciación de ``palabras". Se analizará en el ejercicio 10.

\item descifrarPalabras(?Ls): Si no se instancia Ls, generará infinitas listas. Se recomienda utilizar instanciada con lista de variables.
\end{itemize}
\begin{lstlisting}
descifrar(S, M) :- palabras(S, P),
		   cant_distintos(P, Pn), 	
		   palabras_con_variables(P, V), 
		   descifrarPalabras(V),
		   cant_distintos(V, Pn),
		   juntar_con(V, 32, PalabrasSeparadas), 
		   string_codes(M, PalabrasSeparadas).

descifrarPalabras([]).
descifrarPalabras([Vs | Vss]) :- cant_distintos(Vs, Pn),
			 	 diccionario_lista(Vs), 
				 cant_distintos(Vs, Pn),
				 descifrarPalabras(Vss).
											 
\end{lstlisting}


\subsection{Ejercicio 9}

\begin{itemize}
\item descifrar\_sin\_espacios(+S, ?M), necesariamente S debe estar instanciado para generar las posibles intercalaciones con espacio resultantes, que luego deberan ser descifradas.
Si no estuviera instanciada la S podria instanciarse en secuencias, potencialmente infinitas, que nunca daran una oracion valida considerando el diccionario cargado actual.
\item con\_espacios\_intercalados(+S, ?R)
\item intercalar\_o\_no(+P, +S, ?R)
\end{itemize}
\begin{lstlisting}
descifrar_sin_espacios(S, M) :- con_espacios_intercalados(S, SWithSpaces),  
				descifrar(SWithSpaces, M).

con_espacios_intercalados([], []).
con_espacios_intercalados(S, SWithSpaces) :- 
			append(SPref, SSuf, S), SPref \== [], 
			intercalar_o_no(SPref, SSuf, SPrefIntercalado), 
			con_espacios_intercalados(SSuf, SWithSpacesSuf), 
			append(SPrefIntercalado, SWithSpacesSuf, SWithSpaces).

intercalar_o_no(Pref, [], Pref). 
intercalar_o_no(Pref, _, PrefNuevo) :- append(Pref, [espacio], PrefNuevo).
											 
\end{lstlisting}


\subsection{Ejercicio 10}

\begin{itemize}
\item mensajes\_mas\_parejos(+S, ?M): S debe estar instanciada en particular, por el uso de descifrar\_sin\_espacios(). 
\item hay\_uno\_menor(+DesvioM, +S).
\item calcular\_desvio(+Mensaje, ?Desvio).
\item split\_por\_caracter(+Sentencia, +Caracter, ?Ll).
Sentencia debe estar siempre instanciado debido al uso de "quitar". En caso de no estar instanciado Caracter, el algoritmo termina aunque con funcionamiento extraño.
 Splitea correctamente, sin embargo, Caracter se va unificando con los valores la lista haciendo que se comporte incorrectamente. Se recomienda uso instanciado.
\item generarListas(+Sentencia, ?R): No puede estar instanciado R y no Sentencia. En caso de no estar instanciados ambos, se generan infinitos resultados. 
\item prefixNoVacio(?P, ?L): En caso de estar los dos no instanciados, genera infinitos resultados.
\item sufix(?Prefijo, ?Sentencia, ?Sufijo): misma instanciación que append. En caso de estar todos sin instanciar, se generan infinitos resultados.
\item calcular\_desvio\_sobre\_lista\_de\_palabras(+Palabras, ?Desvio).
\item calcular\_longitud\_media(+P, ?LongMedia).
\item length\_list(+Ls, ?LList).
\item average(+List, ?Average).
\item binomios\_cuadrados(+Ps, +LongMedia, -BCuadrados).
\item binomio\_cuadrado(?P, +LongMedia, -BCuadrado).  
\end{itemize}
\begin{lstlisting}
mensajes_mas_parejos(S, M) :- descifrar_sin_espacios(S, M), 
			      string_codes(M, L),
  			      calcular_desvio(L, DesvioM), 
			      not(hay_uno_menor(DesvioM, S)).

hay_uno_menor(DesvioM, S) :- descifrar_sin_espacios(S, MComparador),
			     string_codes(MComparador, LComparador),
			     calcular_desvio(LComparador, DesvioComp),
			     DesvioM > DesvioComp.

calcular_desvio(Mensaje, Desvio) :- 
	split_por_caracter(Mensaje, 32, MsjSeparadosPorEspacio), 
	calcular_desvio_sobre_lista_de_palabras(MsjSeparadosPorEspacio, Desvio).

split_por_caracter([], _, []).
split_por_caracter(Sentencia, Caracter, Ll) :-
			      quitar(Caracter, Sentencia, SentenciaSinC), 
			      generarListas(SentenciaSinC, Ll),
			      juntar_con(Ll, Caracter, Sentencia).

generarListas([], []).
generarListas(Sentencia, R) :- prefixNoVacio(P, Sentencia), 
			       sufix(P, Sentencia, SentenciaSinP), 
			       generarListas(SentenciaSinP, Rec), 
			       append([P], Rec, R).

prefixNoVacio(P, L) :- append(P, _, L), length(P, Long), Long > 0.

sufix(Prefijo, Sentencia, Sufijo) :- append(Prefijo, Sufijo, Sentencia).

calcular_desvio_sobre_lista_de_palabras(Palabras, Desvio) :- 
		calcular_longitud_media(Palabras, LongMedia), 
		binomios_cuadrados(Palabras, LongMedia, BCuadrados), 
		sum_list(BCuadrados, Sumatoria), 
		length(Palabras, LPalabras), Division is Sumatoria / LPalabras, 
		Desvio is sqrt(Division).

calcular_longitud_media(P, LongMedia) :- length_list(P, LengthList), 
					 average(LengthList, LongMedia).

length_list([L | Ls], LList) :- append([LLength], LListRec, LList), 
				length_list(Ls, LListRec),
				length(L, LLength),!. 
length_list([L], LList) :- length(L, LLength), LList = [LLength].

average(List, Average):- sum_list(List, Sum), 
			 length(List, Length), 
			 Length > 0, 
			 Average is (Sum / Length).


binomios_cuadrados([P], LongMedia, [BCuadrado]) :- 
		binomio_cuadrado(P, LongMedia, BCuadrado). 
binomios_cuadrados([P | Ps], LongMedia, BCuadrados) :- 
		append([BCuadrado], RecBCuadrados, BCuadrados), 
		binomios_cuadrados(Ps, LongMedia, RecBCuadrados), 
		binomio_cuadrado(P, LongMedia, BCuadrado),!.

binomio_cuadrado(P, LongMedia, BCuadrado) :- length(P, LongP), 
					     Resta is (LongP-LongMedia), 
					     BCuadrado is (Resta^2).
											 
\end{lstlisting}


	%~ \newpage
	%~ \bibliographystyle{plain}
	%~ \clearpage
	%~ \bibliography{bibliography}
	%~ \addcontentsline{toc}{section}{Referencias}

\end{document}
