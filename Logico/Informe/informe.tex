\documentclass[spanish, 10pt,a4paper]{article}
\usepackage[spanish]{babel}
\usepackage[utf8]{inputenc}
\usepackage{textcomp}
\usepackage{hyperref}
\usepackage[pdftex]{graphicx}
\usepackage{epsfig}
\usepackage{amsmath}
\usepackage{hyperref}
\usepackage{amssymb}
\usepackage{color}
\usepackage{graphics}
\usepackage{amsthm}
\usepackage{subcaption}
\usepackage{caratula}
\usepackage{fancyhdr,lastpage}
\usepackage[paper=a4paper, left=1.4cm, right=1.4cm, bottom=1.4cm, top=1.4cm]{geometry}
\usepackage[table]{xcolor} % color en las matrices
\usepackage[font=small,labelfont=bf]{caption} % caption de las figuras en letra mas chica que el texto
\usepackage{listings}
\usepackage{float}
\usepackage{pdfpages}
\usepackage{amsfonts}


\color{black}

%%%PAGE LAYOUT%%%
\topmargin = -1.2cm
\voffset = 0cm
\hoffset = 0em
\textwidth = 48em
\textheight = 164 ex
\oddsidemargin = 0.5 em
\parindent = 2 em
\parskip = 3 pt
\footskip = 7ex
\headheight = 20pt
\pagestyle{fancy}
\lhead{IS1 - Trabajo Pr\'actico 1} % cambia la parte izquierda del encabezado
\renewcommand{\sectionmark}[1]{\markboth{#1}{}} % cambia la parte derecha del encabezado
\rfoot{\thepage}
\cfoot{}
\numberwithin{equation}{section} %sets equation numbers <chapter>.<section>.<subsection>.<index>

\newcommand{\figurewidth}{1\textwidth}

\newcommand{\tuple}[1]{\ensuremath{\left \langle #1 \right \rangle }}
\newcommand{\Ode}[1]{\small{$\mathcal{O}(#1)$}}


%El siguiente paquete permite escribir la caratula facilmente
\hypersetup{
  pdftitle={ IS1 - TP1 },
  colorlinks,
  citecolor=black,
  filecolor=black,
  linkcolor=black,
  urlcolor=black 
}

\materia{Paradigmas de Lenguajes}

\titulo{Trabajo Práctico 2}

\subtitulo{Paradigma Lógico.}

\grupo{Grupo Two and a Half Blondes}

\integrante{De Sousa Bispo, Germán}{359/12}{germandesousa@gmail.com}
\integrante{Fernandez, Esteban}{691/12}{esteban.pmf@gmail.com}
\integrante{Wright, Carolina}{876/12}{wright.carolina@gmail.com}

 
\begin{document}
{ \oddsidemargin = 2em
	\headheight = -20pt
	\maketitle
}
	\tableofcontents
	\newpage

\section{Aclaración}
El ejercicio 8 posee errores a la hora de descifrar. Trae repetidos bajo ciertas condiciones y no siempre descifra bien. Se lo anexo para obtener un feedback al respecto para corregirlo posteriormente.

\section{Código}

\subsection{Ejercicio 1}

\begin{itemize}
\item diccionario\_lista(?Lcode): la reversibilidad depende de la reversibilidad de string\_codes, la cual es string\_codes(?String, ?Codes).
\end{itemize}
\begin{lstlisting}
diccionario_lista(Lcode) :- diccionario(PalabraDelDicc), 
			    string_codes(PalabraDelDicc, Lcode). 
\end{lstlisting}

\subsection{Ejercicio 2}

\begin{itemize}
\item juntar\_con(+S, ?J , ?R): Si no se instancia S, el predicado no funciona correctamente (aunque no se cuelga) por la forma en que se realizan los Appends. 
J puede ser una variable que, si R está instanciado, terminará unificándose. Si no, appendeará a la variable J.
\end{itemize}
\begin{lstlisting}
juntar_con([L], _ , L).
juntar_con([Ls | Lss], J, R) :- juntar_con(Lss, J, Rrec), 
				append(RPref, Rrec, R), 
				append(Ls, [J], RPref),!.  
\end{lstlisting}

\subsection{Ejercicio 3}

\begin{itemize}
\item palabras(+S, ?P): Los motivos por los que S debe estar instanciado se muestran en el análisis de split\_por\_caracter. El mismo se encuentra en el ejercicio 10.
\end{itemize}
\begin{lstlisting}
palabras(S, P) :- split_por_caracter(S, espacio, P).  
\end{lstlisting}

\subsection{Ejercicio 4}

\begin{itemize}
\item asignar\_var(?A, +MI, ?MF): Si MI no está instanciado, se cuelga cuando trata de encontrar otra solución más que MI = [] y MF = A, debido a que genera
combinaciones con listas infinitas. A o MF deben estar instanciados. No pueden no estar instanciados al mismo tiempo.

\item get\_keys(?Tuples, ?TMapped): Si ambas no están instanciadas, genera infinitos resultados.

\item Este ejercicio funciona gracias a la capacidad de ProLog de generar variables frescas bajo demanda utilizando 
variables anonimas generadas con la keyword "\_", ademas, la manera de representar dichas variables frescas ayuda para que estas puedan ser manipulables dentro de los predicados.
Si se nos diera como variable fresca una que ya hemos utilizado antes nuestro predicado asignar\_var se volveria inconsistente.
\end{itemize}
\begin{lstlisting}

asignar_var(A, MI, MF) :- get_keys(MI, Keys), 
			  not(member(A, Keys)), 
			  append(MI, [(A, _)], MF).
asignar_var(A, MI, MI) :- get_keys(MI, Keys), 
			  member(A, Keys).

get_keys([], []).
get_keys([(Key, _) | Ts], TMapped) :- get_keys(Ts, Trec), 
				      append([Key], Trec, TMapped).
\end{lstlisting}

\subsection{Ejercicio 5}

\begin{itemize}
\item palabras\_con\_variables(+L, ?V): Si L no estuviera instanciado, no se rompe pero genera infinitos resultados inservibles con combinaciones de listas vacias. Esto sucede debido a la utilización de member().
\item palabras\_con\_var\_y\_mapa(+L, ?V, +Mi, -Mf)
\item pares\_definidos\_en\_mapa(+L, ?V, +Mi, -Mf)
\end{itemize}
\begin{lstlisting}
palabras_con_variables(L, V) :- palabras_con_var_y_mapa(L, V, [], _).

palabras_con_var_y_mapa([], [], Mf, Mf).
palabras_con_var_y_mapa([Ls | Lss], [Vs | Vss], M, Mf1) :- 
				pares_definidos_en_mapa(Ls, Vs, M, Mf0), 
				palabras_con_var_y_mapa(Lss, Vss, Mf0, Mf1).

pares_definidos_en_mapa([], [], Mf, Mf).
pares_definidos_en_mapa([X | Ls], [V | Vs], Mi0, Mf) :-
				asignar_var(X, Mi0 , Mi1), 
				member((X,V), Mi1), 
				pares_definidos_en_mapa(Ls, Vs, Mi1, Mf).
\end{lstlisting}


\subsection{Ejercicio 6}

\begin{itemize}
\item quitar(?E, +Ls, ?R): E puede no estar instanciado ya que el algoritmo también elimina variables en caso de estar. Ls deberá estar instanciada para evitar la generación de 
listas infinitas. Si no esta instanciado, devuelve una solución trivial y luego se cuelga cuando se pide otra (debido al uso de append).
\end{itemize}
\begin{lstlisting}
quitar(_, [], R) :- R = [].
quitar(E, [L|Ls], R) :- E == L, quitar(E, Ls, R).
quitar(E, [L|Ls], R) :- E \== L, quitar(E, Ls, Rrec), 
			append([L], Rrec, R).
\end{lstlisting}


\subsection{Ejercicio 7}

\begin{itemize}
\item cant\_distintos(+Ls, ?S): Ls deberá estar instanciado para evitar la generación de listas infinitas. Si no esta instanciado, devuelve una solución trivial y luego se cuelga 
cuando se pide otra (debido al uso de quitar(), quien termina llamando a append()).
\end{itemize}
\begin{lstlisting}
cant_distintos([], S) :- S = 0.
cant_distintos([L | Ls], S) :- quitar(L, Ls, RQuitado), 
			       cant_distintos(RQuitado, CuentaRec), 
			        S is (1 + CuentaRec).
\end{lstlisting}


\subsection{Ejercicio 8}

\begin{itemize}
\item Aclaración: este ejercicio devuelve incorrectamente repetidos y soluciones inválidas.
\end{itemize}
\begin{lstlisting}
descifrar(S, M) :- palabras(S, P), 	
		   palabras_con_variables(P, V), 
    		   descifrarPalabras(V, Mvar), 
		   juntar_con(Mvar, 32, PalabrasSeparadas), 
		   string_codes(M, PalabrasSeparadas).

descifrarPalabras([], []).
descifrarPalabras([Vs | Vss], Mvar) :- diccionario_lista(PalabraDelDicc), 
					palabra_valida(Vs, PalabraDelDicc, Mp), 
					descifrarPalabras(Vss, Mrec), 
					append([Mp], Mrec, Mvar).

palabra_valida([], [], []).
palabra_valida([Var | Vars], [P | Ps], M) :- length(Vars, Lv), 
					     length(Ps, Lp), 
  					     Lv == Lp, 
					     palabra_valida(Vars, Ps, Mrec), 
					     esta_libre(Var, P, Ps),
					     P = Var, 
					     append([P], Mrec, M).

esta_libre(_, _, []).										 
esta_libre(Var, _, _) :- nonvar(Var).											 
esta_libre(Var, P, [Pp | Ps]) :- var(Var), Pp \== P, esta_libre(Var, P, Ps).
											 
\end{lstlisting}


\subsection{Ejercicio 9}

\begin{itemize}
\item descifrar\_sin\_espacios(+S, ?M), necesariamente S debe estar instanciado para generar las posibles intercalaciones con espacio resultantes, que luego deberan ser descifradas.
Si no estuviera instanciada la S podria instanciarse en secuencias, potencialmente infinitas, que nunca daran una oracion valida considerando el diccionario cargado actual.
\item con\_espacios\_intercalados(+S, ?R)
\item intercalar\_o\_no(+P, +S, ?R)
\end{itemize}
\begin{lstlisting}
descifrar_sin_espacios(S, M) :- con_espacios_intercalados(S, SWithSpaces),  
				descifrar(SWithSpaces, M).

con_espacios_intercalados([], []).
con_espacios_intercalados(S, SWithSpaces) :- 
			append(SPref, SSuf, S), SPref \== [], 
			intercalar_o_no(SPref, SSuf, SPrefIntercalado), 
			con_espacios_intercalados(SSuf, SWithSpacesSuf), 
			append(SPrefIntercalado, SWithSpacesSuf, SWithSpaces).

intercalar_o_no(Pref, [], Pref) :- !. 
intercalar_o_no(Pref, _, PrefNuevo) :- append(Pref, [espacio], PrefNuevo).
											 
\end{lstlisting}


\subsection{Ejercicio 10}

\begin{itemize}
\item mensajes\_mas\_parejos(+S, ?M): S debe estar instanciada en particular, por el uso de descifrar\_sin\_espacios(). 
\item hay\_uno\_menor(+DesvioM, +S).
\item calcular\_desvio(+Mensaje, ?Desvio).
\item split\_por\_caracter(?Sentencia, +Caracter, ?ListaDePalabras): En caso de que ListaDePalabras esté instanciado, Sentencia debe estar instanciado.
\item leer\_hasta\_caracter(+Caracteres, +CaracterSeparador, ?Palabra).
\item palabra\_hasta\_caracter(+Cs, +CaracterSeparador, ?Accum, ?Palabra): En caso de instanciarse Accum, debe tener relacion con lo puesto en Cs para que el algoritmo tenga sentido. Hay casos para los que funciona tener ?Cs, pero en otros se cuelga. Al igual que para Accum, esos casos deben tener sentido en el algoritmo. No se recomienda.
\item borrar\_hasta\_caracter(?Palabra, ?Caracter, ?Sentencia, ?SentenciaSinPalabra): Siempre brinda una sola solución. 
\item calcular\_desvio\_sobre\_lista\_de\_palabras(+Palabras, ?Desvio).
\item calcular\_longitud\_media(+P, ?LongMedia).
\item length\_list(+Ls, ?LList).
\item average(+List, ?Average).
\item binomios\_cuadrados(+Ps, +LongMedia, -BCuadrados).
\item binomio\_cuadrado(?P, +LongMedia, -BCuadrado).  
\end{itemize}
\begin{lstlisting}
mensajes_mas_parejos(S, M) :- descifrar_sin_espacios(S, M), 
			      string_codes(M, L),
  			      calcular_desvio(L, DesvioM), 
			      not(hay_uno_menor(DesvioM, S)).

hay_uno_menor(DesvioM, S) :- descifrar_sin_espacios(S, MComparador),
			     string_codes(MComparador, LComparador),
			     calcular_desvio(LComparador, DesvioComp),
			     DesvioM > DesvioComp.

calcular_desvio(Mensaje, Desvio) :- 
	split_por_caracter(Mensaje, 32, MsjSeparadosPorEspacio), 
	calcular_desvio_sobre_lista_de_palabras(MsjSeparadosPorEspacio, Desvio).

split_por_caracter([], _, []).
split_por_caracter(Sentencia, Caracter, ListaDePalabras) :- 
		leer_hasta_caracter(Sentencia, Caracter, Palabra),
		borrar_hasta_caracter(Palabra, Caracter, Sentencia, SSinPalabra),
		split_por_caracter(SSinPalabra, Caracter, RecListaDePalabras),
		append([Palabra], RecListaDePalabras, ListaDePalabras), !.

leer_hasta_caracter(Caracteres, CaracterSeparador, Palabra) :- 
		palabra_hasta_caracter(Caracteres, CaracterSeparador, [], Palabra).

palabra_hasta_caracter([], _, Palabra, Palabra).
palabra_hasta_caracter([C|Cs], CaracterSeparador, Accum, Palabra) :- 
	C\=CaracterSeparador,
	append(Accum, [C], AccumConCaracter),
	palabra_hasta_caracter(Cs, CaracterSeparador, AccumConCaracter, Palabra).
palabra_hasta_caracter([C|_], CaracterSeparador, Palabra, Palabra) :- 
	C==CaracterSeparador.

borrar_hasta_caracter(Palabra, Caracter, Sentencia, SentenciaSinPalabra) :- 
		append(Palabra, [Caracter|SentenciaSinPalabra], Sentencia), !.
borrar_hasta_caracter(Palabra, _, Sentencia, SentenciaSinPalabra) :- 
		append(Palabra, SentenciaSinPalabra, Sentencia).

calcular_desvio_sobre_lista_de_palabras(Palabras, Desvio) :- 
		calcular_longitud_media(Palabras, LongMedia), 
		binomios_cuadrados(Palabras, LongMedia, BCuadrados), 
		sum_list(BCuadrados, Sumatoria), 
		length(Palabras, LPalabras), Division is Sumatoria / LPalabras, 
		Desvio is sqrt(Division).

calcular_longitud_media(P, LongMedia) :- length_list(P, LengthList), 
					 average(LengthList, LongMedia).

length_list([L | Ls], LList) :- append([LLength], LListRec, LList), 
				length_list(Ls, LListRec),
				length(L, LLength),!. 
length_list([L], LList) :- length(L, LLength), LList = [LLength].

average(List, Average):- sum_list(List, Sum), 
			 length(List, Length), 
			 Length > 0, 
			 Average is (Sum / Length).


binomios_cuadrados([P], LongMedia, [BCuadrado]) :- 
		binomio_cuadrado(P, LongMedia, BCuadrado). 
binomios_cuadrados([P | Ps], LongMedia, BCuadrados) :- 
		append([BCuadrado], RecBCuadrados, BCuadrados), 
		binomios_cuadrados(Ps, LongMedia, RecBCuadrados), 
		binomio_cuadrado(P, LongMedia, BCuadrado),!.

binomio_cuadrado(P, LongMedia, BCuadrado) :- length(P, LongP), 
					     Resta is (LongP-LongMedia), 
					     BCuadrado is (Resta^2).
											 
\end{lstlisting}


	%~ \newpage
	%~ \bibliographystyle{plain}
	%~ \clearpage
	%~ \bibliography{bibliography}
	%~ \addcontentsline{toc}{section}{Referencias}

\end{document}
