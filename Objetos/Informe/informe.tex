\documentclass[spanish, 10pt,a4paper]{article}
\usepackage[spanish]{babel}
\usepackage[utf8]{inputenc}
\usepackage{textcomp}
\usepackage{hyperref}
\usepackage[pdftex]{graphicx}
\usepackage{epsfig}
\usepackage{amsmath}
\usepackage{hyperref}
\usepackage{amssymb}
\usepackage{color}
\usepackage{graphics}
\usepackage{amsthm}
\usepackage{subcaption}
\usepackage{caratula}
\usepackage{fancyhdr,lastpage}
\usepackage[paper=a4paper, left=1.4cm, right=1.4cm, bottom=1.4cm, top=1.4cm]{geometry}
\usepackage[table]{xcolor} % color en las matrices
\usepackage[font=small,labelfont=bf]{caption} % caption de las figuras en letra mas chica que el texto
\usepackage{listings}
\usepackage{float}
\usepackage{pdfpages}
\usepackage{amsfonts}


\color{black}

%%%PAGE LAYOUT%%%
\topmargin = -1.2cm
\voffset = 0cm
\hoffset = 0em
\textwidth = 48em
\textheight = 164 ex
\oddsidemargin = 0.5 em
\parindent = 2 em
\parskip = 3 pt
\footskip = 7ex
\headheight = 20pt
\pagestyle{fancy}
\lhead{PLP - Trabajo Pr\'actico 3} % cambia la parte izquierda del encabezado
\renewcommand{\sectionmark}[1]{\markboth{#1}{}} % cambia la parte derecha del encabezado
\rfoot{\thepage}
\cfoot{}
\numberwithin{equation}{section} %sets equation numbers <chapter>.<section>.<subsection>.<index>

\newcommand{\figurewidth}{1\textwidth}

\newcommand{\tuple}[1]{\ensuremath{\left \langle #1 \right \rangle }}
\newcommand{\Ode}[1]{\small{$\mathcal{O}(#1)$}}


%El siguiente paquete permite escribir la caratula facilmente
\hypersetup{
  pdftitle={ IS1 - TP1 },
  colorlinks,
  citecolor=black,
  filecolor=black,
  linkcolor=black,
  urlcolor=black 
}

\materia{Paradigmas de Lenguajes}

\titulo{Trabajo Práctico 2}

\subtitulo{Paradigma Lógico.}

\grupo{Grupo Two and a Half Blondes}

\integrante{De Sousa Bispo, Germán}{359/12}{germandesousa@gmail.com}
\integrante{Fernandez, Esteban}{691/12}{esteban.pmf@gmail.com}
\integrante{Wright, Carolina}{876/12}{wright.carolina@gmail.com}

 
\begin{document}
{ \oddsidemargin = 2em
	\headheight = -20pt
	\maketitle
}
	\tableofcontents
	\newpage

\section{Código}

\subsection{PropositionalFormula}

\begin{lstlisting}

Object subclass: #PropositionalFormula
	instanceVariableNames: ''
	classVariableNames: ''
	poolDictionaries: ''
	category: 'PLP-TP3'!

\end{lstlisting}

\subsubsection{Métodos de instancia}
\begin{lstlisting}
not 
  ^ Negation of: self.

& aFormula
  ^ Conjunction of: self and: aFormula.

| aFormula
  ^ Disjunction of: self and: aFormula.

==> aFormula
  ^ Implication of: self and: aFormula.

= aFormula 
  ^ SubclassResponsibility

hash 
  ^ SubclassResponsibility

allPropVars
  ^ SubclassResponsibility 

negate
  ^ SubclassResponsibility

toNNF
  | formulaWithoutImplications |
  formulaWithoutImplications := self withoutImplications.
  ^ formulaWithoutImplications organizeNegations.

withoutImplications
  ^ SubclassResponsibility

organizeNegations
  ^ SubclassResponsibility

organizeNegationsFromNegation: aNegationFormula
  ^ SubclassResponsibility

operatorAsString
  ^ SubclassResponsibility

asString
  ^ SubclassResponsibility

printString
  ^ self asString

asStringWithParenthesis: aFormula
  ^ '( ', aFormula asString, ' )'.

asStringWithoutParenthesis: aFormula
  ^ aFormula asString.

representationAsStringIn: aBinaryFormula
  ^ SubclassResponsibility.

representationAsStringInNegation: aNegationFormula
  ^ aNegationFormula asStringWithParenthesis: self.

\end{lstlisting}

\subsubsection{Métodos de Clase}
\begin{lstlisting}
PropositionalFormula class
  instanceVariableNames: ''

of: aFormula and: anotherFormula
  ^ SubclassResponsibility.

of: aFormula
  ^ SubclassResponsibility 

\end{lstlisting}

\subsection{BinaryPropositionalFormula}

\begin{lstlisting}

PropositionalFormula subclass: #BinaryPropositionalFormula
	instanceVariableNames: 'firstFormula secondFormula operator'
	classVariableNames: ''
	poolDictionaries: ''
	category: 'PLP-TP3'

\end{lstlisting}

\subsubsection{Métodos de Instancia}
\begin{lstlisting}
initWith: aFormula and: anotherFormula
  firstFormula := aFormula. 
  secondFormula := anotherFormula.

setOperator: aOperator
  operator := aOperator

value: aValuation
  |firstResult secondResult msg |
  firstResult := firstFormula value: aValuation.
  secondResult := secondFormula value: aValuation.
  msg := Message selector: operator argument: secondResult.
  ^ msg sendTo: firstResult.

= aFormula
  "Had to implement this short-circuit evaluation. 
   If not, should have implemented my own Boolean
   If didn't do this, UnaryPropositionalFormula and PropositionalVariables 
   would have fail to understand secondFormula message
   It's not their responsibility to even know that secondFormula message exists"

  (self class == aFormula class) ifFalse: [ ^ false ].
  ^ (firstFormula = (aFormula firstFormula)) and: 
	(secondFormula = (aFormula secondFormula))

hash 
  ^ self class hash + firstFormula hash + secondFormula hash

firstFormula
  ^ firstFormula.

secondFormula
  ^ secondFormula.

allPropVars
  ^ firstFormula allPropVars union: secondFormula allPropVars.

negate
  ^ SubclassResponsibility

withoutImplications
  |firstResult secondResult msg |
  firstResult := firstFormula withoutImplications. 
  secondResult := secondFormula withoutImplications.
  msg := Message selector: operator argument: secondResult.
  ^ msg sendTo: firstResult.

organizeNegations
  |firstResult secondResult msg |
  firstResult := firstFormula organizeNegations.
  secondResult := secondFormula organizeNegations.
  msg := Message selector: operator argument: secondResult.
  ^ msg sendTo: firstResult.

organizeNegationsFromNegation: aNegationFormula
  ^ aNegationFormula organizeByNegating: self. 

asString
  | firstFormulaAsString secondFormulaAsString |
  firstFormulaAsString := firstFormula representationAsStringIn: self.
  secondFormulaAsString := secondFormula representationAsStringIn: self.
  ^ firstFormulaAsString, self operatorAsString, secondFormulaAsString

representationAsStringIn: aBinaryFormula
  ^ aBinaryFormula asStringWithParenthesis: self.

\end{lstlisting}

\subsubsection{Métodos de Clase}
\begin{lstlisting}
BinaryPropositionalFormula class
  instanceVariableNames: ''

of: aFormula and: anotherFormula
  ^ self new initWith: aFormula and: anotherFormula

of: aFormula
  ^ ShouldNotImplement 

\end{lstlisting}

\subsection{Conjunction}

\begin{lstlisting}
BinaryPropositionalFormula subclass: #Conjunction
	instanceVariableNames: ''
	classVariableNames: ''
	poolDictionaries: ''
	category: 'PLP-TP3'
\end{lstlisting}

\subsubsection{Métodos de Instancia}
\begin{lstlisting}
operatorAsString
  ^ ' & '

negate
  ^ firstFormula negate | (secondFormula negate).

\end{lstlisting}

\subsubsection{Métodos de Clase}

\begin{lstlisting}

Conjunction class
	instanceVariableNames: ''

of: aFormula and: anotherFormula
	| formula |
	formula := super of: aFormula and: anotherFormula.
	^ formula setOperator: #&.
\end{lstlisting}

\subsection{Disjunction}

\begin{lstlisting}

BinaryPropositionalFormula subclass: #Disjunction
	instanceVariableNames: ''
	classVariableNames: ''
	poolDictionaries: ''
	category: 'PLP-TP3'
\end{lstlisting}

\subsubsection{Métodos de Instancia}
\begin{lstlisting}

operatorAsString
  ^ ' | '

negate
  ^ firstFormula negate & (secondFormula negate).

\end{lstlisting}

\subsubsection{Métodos de Clase}

\begin{lstlisting}
Disjunction class
	instanceVariableNames: ''

of: aFormula and: anotherFormula
	| formula |
	formula := super of: aFormula and: anotherFormula.
	^ formula setOperator: #|.

\end{lstlisting}

\subsection{Implication}

\begin{lstlisting}

BinaryPropositionalFormula subclass: #Implication
	instanceVariableNames: ''
	classVariableNames: ''
	poolDictionaries: ''
	category: 'PLP-TP3'

\end{lstlisting}

\subsubsection{Métodos de Instancia}
\begin{lstlisting}

operatorAsString
  ^ ' ==> '

negate
  ^ firstFormula & (secondFormula negate).

withoutImplications
  ^ (firstFormula not | secondFormula) withoutImplications

organizeNegations
  ^ ShouldNotImplement. 

organizeNegationsFromNegation: aNegationFormula
  ^ ShouldNotImplement.

\end{lstlisting}

\subsubsection{Métodos de Clase}

\begin{lstlisting}
Implication class
	instanceVariableNames: ''

of: aFormula and: anotherFormula
  | formula |
  formula := super of: aFormula and: anotherFormula.
  ^ formula setOperator: #==>.

\end{lstlisting}

\subsection{UnaryPropositionalFormula}

\begin{lstlisting}

PropositionalFormula subclass: #UnaryPropositionalFormula
	instanceVariableNames: 'firstFormula'
	classVariableNames: ''
	poolDictionaries: ''
	category: 'PLP-TP3'
\end{lstlisting}

\subsubsection{Métodos de Instancia}

\begin{lstlisting}
initWith: aFormula
  firstFormula := aFormula.

value: aValuation
  ^ SubclassResponsibility

= aFormula
  "Had to implement this short-circuit evaluation. 
   If not, should have implemented my own Boolean
   If didn't do this, UnaryPropositionalFormula and PropositionalVariables 
   would have fail to understand secondFormula message
   It's not their responsibility to even know that secondFormula message exists"
  (self class == aFormula class) ifFalse: [ ^ false ].
  ^ (firstFormula = (aFormula firstFormula))

hash 
  ^ self class hash + firstFormula hash

firstFormula
  ^ firstFormula.

allPropVars
  ^ firstFormula allPropVars.

negate
  ^ SubclassResponsibility

asString
  | formulaAsString |
  formulaAsString := firstFormula representationAsStringInNegation: self.
  ^ self operatorAsString, formulaAsString

representationAsStringIn: aBinaryFormula
  ^ aBinaryFormula asStringWithoutParenthesis: self.

\end{lstlisting}

\subsubsection{Métodos de Clase}
\begin{lstlisting}
 
UnaryPropositionalFormula class
  instanceVariableNames: ''

of: aFormula and: anotherFormula
  ^ ShouldNotImplement

of: aFormula
  ^ self new initWith: aFormula
\end{lstlisting}

\subsection{Negation}

\begin{lstlisting}

UnaryPropositionalFormula subclass: #Negation
	instanceVariableNames: ''
	classVariableNames: ''
	poolDictionaries: ''
	category: 'PLP-TP3'

\end{lstlisting}

\subsubsection{Métodos de Instancia}
\begin{lstlisting}
value: aValuation
  ^ (firstFormula value: aValuation) not

operatorAsString
  El simbolo de negacion no anda en latex con los plugins usados.

negate
  ^ firstFormula.

withoutImplications 
  ^ firstFormula withoutImplications not

organizeNegations
  ^ firstFormula organizeNegationsFromNegation: self.

organizeNegationsFromNegation: aNegationFormula
  ^ aNegationFormula organizeByNegating: self.

notOrganize: aFormula
  ^ self.

organizeByNegating: aFormula
  ^ aFormula negate organizeNegations.


\end{lstlisting}

\subsubsection{Métodos de Clase}

\begin{lstlisting}
Negation class
	instanceVariableNames: ''

of: aFormula and: anotherFormula
	^ MessageNotUnderstood.

\end{lstlisting}

\subsection{PropositionalVariable}

\begin{lstlisting}

Object subclass: #PropositionalVariable
	instanceVariableNames: 'name'
	classVariableNames: ''
	poolDictionaries: ''
	category: 'PLP-TP3'


\end{lstlisting}

\subsubsection{Métodos de Instancia}
\begin{lstlisting}
not 
  ^ Negation of: self

& aFormula
  ^ Conjunction of: self and: aFormula

| aFormula
  ^ Disjunction of: self and: aFormula

==> aFormula
  ^ Implication of: self and: aFormula

initWith: aName
  name := aName.

value: aValuation
  ^ aValuation includes: name.

= aFormula
  ^ (self class == aFormula class) and: (name = (aFormula name))

allPropVars
  ^ Set with: name 

negate
  ^ self not.

withoutImplications 
  ^ self

organizeNegations
  ^ self

organizeNegationsFromNegation: aNegationFormula
  ^ aNegationFormula notOrganize: self.

asString
  ^ name

printString
  ^ self asString

representationAsStringIn: aBinaryFormula
  ^ aBinaryFormula asStringWithoutParenthesis: self.

representationAsStringInNegation: aNegationFormula
  ^ aNegationFormula asStringWithoutParenthesis: self.

\end{lstlisting}
\subsubsection{Métodos de Clase}

\begin{lstlisting}
PropositionalVariable class
	instanceVariableNames: ''

named: aName
	^ self new initWith: aName.

 \end{lstlisting}
	%~ \newpage
	%~ \bibliographystyle{plain}
	%~ \clearpage
	%~ \bibliography{bibliography}
	%~ \addcontentsline{toc}{section}{Referencias}

\end{document}
