\section{Introducción}

	Para este Trabajo Práctico, se nos pidió especificar un sistema para reemplazar el Sistema Electoral Nacional actual. El objetivo de este nuevo sistema es volverlo más moderno, ágil y transparente. Dado que el formato de la votación no presenta cambios, el sistema debe proveer las mismas modalidades de voto que el sistema actual; es decir, el sistema debe permitir el voto de una lista completa, el voto por categoría (o `corte de boleta') y el voto en blanco.
	
	Para este objetivo, se nos pide diseñar el comportamiento de una máquina emisora de sufragios, a razón de una por cada mesa de votación, la que permita a los Electores efectuar el sufragio de una forma clara, transparente y secreta.
	
	A su vez, nuestro Sistema debe especificar el comportamiento del Centro de Cómputos Nacional, encargado de recibir los resultados desde todas las máquinas y procesar estos resultados, dando a conocer los resultados mientras las mesas se escrutan.
	
	En particular, nuestro sistema debe implementar las siguientes funcionalidades:
\begin{itemize}
\item Poder cargar y consultar el padrón electoral.
\item Poder cargar y consultar los candidatos disponibles para cada cargo.
\item Poder asignar al Presidente de Mesa para cada una de las mesas.
\item Pasadas las 18 horas, poder registrar y consultar los Resultados parciales y totales.
\item Poder calcular los candidatos electos para los cargos ejecutivos y legistlativos.
\item Poder cargar en las máquinas los candidatos antes del día electoral.
\item El Presidente de Mesa debe poder abrir y cerrar la mesa en el día electoral.
\item Poder garantizar al Elector la posibilidad de efectuar su sufragio en forma secreta.
\item Poder dar asistencia al Elector para el uso de las máquinas de sufragio, incluso en el caso de votantes no videntes o parcialmente discapacitados.
\item Poder dar a los fiscales la posibilidad de controlar los comicios.
\end{itemize}

	También, debemos especificar de que forma nuestro sistema a especificar almacenará los sufragios de los Electores. Debemos decidir entre varias alternativas: boleta impresa, digital en un medio de almacenamiento a designar, etc.
	
	En este informe, utilizamos diferentes técnicas para especificar un Sistema Electoral Nacional, tales como diagramas de contexto, actividad y clases, máquinas de estados finitos.
